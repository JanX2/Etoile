\documentclass[11pt]{report}
\usepackage{geometry}                % See geometry.pdf to learn the layout options. There are lots.
\geometry{a4paper}                   % ... or a4paper or a5paper or ... 
%\geometry{landscape}                % Activate for for rotated page geometry
\usepackage[parfill]{parskip}    % Activate to begin paragraphs with an empty line rather than an indent
\usepackage{graphicx}
\usepackage{amssymb}
\usepackage{epstopdf}
\usepackage{xspace}
\newcommand{\etoile}{\'Etoil\'e\xspace}
\newcommand{\proman}{Project Manager\xspace}
\newcommand{\gnustep}{GNUstep\xspace}
\DeclareGraphicsRule{.tif}{png}{.png}{`convert #1 `dirname #1`/`basename #1 .tif`.png}

\title{Project Manager Notes}
\author{Christopher Armstrong (carmstrong@fastmail.com.au)}
%\date{}                                           % Activate to display a given date or no date

\begin{document}
\maketitle
\tableofcontents
\section{Abstract}
\proman is an \etoile service for managing the user's workspace. This is a draft set of notes about the concepts underpinning \proman and the way a user might interact with it. Furthermore, this document proposes an architectural model for how \proman might integrate with other \etoile components (such as EtoileUI or CoreObject) in order to realise these use cases.

This document is not an official guide to \etoile concepts, imagined usage scenarios or architecture. Instead, its a memory dump on the part of its author to try and grapple with the ideas set forth by various persons working on the project and try and imagine a concrete implementation of them.

\chapter{Overview}
\section{Terminology}
\begin{description}
\item[Active project] The open project that is currently selected in the user interface and which is displaying its open views. There is one active project at a time.
\item[Default View] The default view of an object that is displayed when it is opened and presented visually in an Object Manager view.
\item[Document] A document is a top-level object in the \etoile workspace. It belongs to a project, and may be represented by multiple views in the workspace. Documents can be created by the \proman or through some action in another application.
\item[Inactive project] A project that is open and running but is not currently selected. 
\item[Object] A data model in the workspace. Objects are not represented directly. Instead, they may have one or more representations in the form of a view.
\item[Object Manager] A service within \proman that lets a user manage the objects within their workspace.
\item[Open project] A project that has all its views opened and running.
\item[Overlay] A window that is temporarily displayed over the top of the workspace.
\item[Private Service] A service that is not under the direct control of the user but that provides a underlying system function. Might also be called a System Service.
\item[Project]  A project is a grouping for a set of related objects, including views, contacts, documents, discussions, music, video, lists and other pieces. The user adds and removes objects invidually from a project - it is not defined by filters or tags.
\item[Project Overlay] A overlay window that shows a version of the Object Manager filtered to the current project by default.
\item[Project Switcher] An onscreen toolbar for switching between projects and closing active projects.
\item[Service] A tool or application in the \etoile environment.
\item[Tag] A user-created search on the objects in the workspace.
\item[User Service] A service is a GNUstep application that displays the contents of documents in views on the screen. A service has one instance in the user's workspace, but may be displaying multiple documents and views.
\item[View] A view is a representation of an object in the workspace. Most objects have only one (primary) view, but there may be more than one view that represents the same object. From the user's perspective, a view is a window on the screen. They are called views in \etoile because they are more than just windows: a view in Etoile remembers where on the screen it was placed, the internal layout and the data being edited.
\end{description}

\section{Introduction}
Project Manager is the \etoile service for managing your workspace.

\chapter{Use Cases}
This chapter highlights some high-level use cases that are used to make decisions about how the environment should function.

\section{Use Case: Creating a new playlist}
Music files are top-level objects in the environment and can be manipulated as such. The user switches to the project overlay and opens a playlist that they were listening to the last time they logged in. The playlist view appears, with a set of controls to play and stop the music at the top of the view, and a list of songs at the bottom.

The user wants to build a new playlist, so they start by going to the document menu and selecting Create New Playlist. The system creates a new playlist and displays it in a new view. They user drags some songs from the existing playlist they just opened, but they want to add some music from their music library too. In order to do this, they create a new Object Manager view and filter it to show all of their music, including playlists, albums, artists and individual songs.

\chapter{Interactions}

\section{Workspace Commands}

\subsection{Create new project}
The user creates a new project by selecting an option from within the \proman menus or the project switcher. The user gives the project a title, and then, \proman will switch to the project.

\subsection{Open a project}
The user clicks the projects icon and the projects overlay appears, displaying all the inactive projects. The user double clicks on one of the projects to open it. \proman opens the project and switches to it.

\subsection{Switch to an active project}
The user switches to an open project by clicking on its icon in project switcher. Before switching, \proman will hide all the views of the currently active project. \proman will then restore all the active views of the selected project and make the project's icon active on the project switcher.

\subsection{Close a project}
The user holds their mouse over the representation of the project in the project switcher until the close button appears. They then click the close button to make the project close.

Alternatively, the user selects Close Project from the Project menu.

All of its open views will be closed and the project representation will be removed from the project switcher. If this was the active project, \proman will switch to the last active project.

\subsection{Create a new view}
A new view can materialise in a few ways:
\begin{itemize}
\item The user goes to the Document menu and selects a new document type to create.
\item The user drags or copies some content out of an existing view into the project so such that it is linked or copied into a new view.
\item The user goes to the template chooser and selects a template to instantiate as a new document.
\end{itemize}

A view isn't necessarily the same as a document. For example, a user may create a new document using the template chooser, but the service creating the document may spawn two views to show the document. Each view will have a unique view ID, but when the service that runs the view is asked to restore it, the data stored under that view identifier is will contain a reference to the document data that it contains. Both views will contain a reference to the same document, but it is up to the service to load the document only once.

\subsection{Search}

The user opens the project overlay. In the search box, they begin typing some text that they wish to search for in their workspace. The Object Manager shows a list of results that are filtered for the current project.

In another scenario, the user creates a new Object Manager view. They click on the Documents type tag, which searches all the documents in their workspace.

\subsection{Open a view within the project}
The user double-clicks on an empty portion of the desktop to open the Project Overlay. Inside the project overlay is the views that are part of the current project but are not currently open. The user double-clicks on a view and it drops down to the desktop and open. 

At this point, the user can open another view by double-clicking on it, but they instead close the Project Overlay by just clicking outside of its overlay window.

\subsection{Switch between open views}
The \proman user interface could provide a few ways to select and switch between open views:
\begin{itemize}
\item a shortcut key combination to make each view in the stack active each time it is pressed
\item an overlay that displays the active views in the current project which the user can select between
\end{itemize}

\subsection{Close a view}
The user clicks the close button on the active view. The view disappears from the screen, but is still available in the project overlay to be opened again.

\subsection{Minimise or restore a view}
The user clicks the minimise button on the active view

\chapter{Conceptual Architecture}

The runtime architecture of Project Manager is split into two parts. The first part is a window manager, which offers tight integration with \etoile and GNUstep to integrate both closely in an X11 environment. The second part is a private service that displays the user interface used for managing projects and views and facilitates some auxiliary environment tasks.

\section{Concepts}
\subsection{Project}
A project is a container for a set of related objects and views. From the user's perspective, switching between projects behaves something like virtual desktops, but the metaphor is extended so that the object manager and the project overlay are filtered to show the objects/views from the active project.

There is always one active project, one or more open projects and any number of unopen or closed projects. An open project is a project that has been loaded into memory and where some of its views are running. An unopen project is where all the views and the project itself are unloaded and remain on disk. The active project is the currently selected project which is displaying its open views on screen. Inactive projects, i.e. all the open projects except the active project, run in the background with their windows hidden until the project is made active.

\subsection{Object}
An object has a more specific meaning in \proman than its looser definition in other architectures. At the most abstract level, it is a living data model in the workspace.

It can be running or in a persisted state. A persisted object lives as a serialised graph of data in an object bundle accessed through CoreObject. It has a unique identifier (known as a UUID) that can be used to obtain it at any point in a user's workspace.

Objects in all object-oriented system have attributes, but in \etoile, we isolate certain attributes and make them indexable as meta-data. Each property has a value which is stored in a meta-data database that forms the basis for searching and filtering on objects in an Object Manager view.

\subsection{Views}
The concept of a view in \etoile, and especially in Project Manager, is the ability for the environment to remember the placement and internal layout of a window. The responsibility for remembering this is split between different parts of the environment. A view is a visual representation of an object in the environment, and an object can have multiple views.

From the user's perspective, a view will be the main way they interact with an object. Each object shall have a primary view which is used when the object is selected for opening from the object manager, and when the view is manipulated in the environment from an object's perspective (such as dragging and dropping a view to another location). While the environment should support multiple views per object, there are only a few use cases where this is very useful.

In terms of the operating system, a view is essentially a window. From a visual perspective, \proman is responsible for remembering everything from the window border outwards. This means that it will remember the position of the window on the screen, where it appeared in the z-order, the width and height, and whether it was running, minimised or closed. Everything internal to the window such as the document being edited and the layout of user interface items is the responsibility of the service displaying the window. This will be provided by the EtoileUI framework.

A view is not actually created by \proman even if it initiates the process of instantiating a new view. Instead, an \etoile service will create a view in response to some kind of user command. This means that \proman is not responsible for loading the view, but only tracking when it is created or restored, moved around on the screen and when it is closed. So that \proman can track the window corresponding to that view, the view needs to uniquely identify itself. Using the identifier, we can easily track when a view is mapped and when it is unmapped. The process of restoring views and tracking them on the screen is described in section \ref{sec:x11_integration} (p.\pageref{sec:x11_integration}) on X11 integration.

\subsection{Services}
Each service is a GNUstep application or tool. We use the name `service' to distinguish them from ordinary applications because users don't interact with them in the traditional sense of an application. Instead of starting up an application as a direct user action, the user does something in the environment like creating a new document or opening a view that might cause a service to begin running. A service doesn't have to interact with the user either: some services provide a background function such as converting data into another format which doesn't require user interaction (much along the lines of the traditional concept of a \gnustep service).

Because a service is a GNUstep application or a tool, and in turn a programme, the lifetime of a service is just that of an ordinary application i.e. it is started as a new process, it can be tracked using its process identifier, we can communicate with it using inter-process communication primitives like sockets and shared memory, and it will die when it is closed. Almost every operating system support these fundamental ideas in essentially the same way that GNUstep is portable and can maintain the same Application Programming Interface on each OS without much difficulty.  

\subsection{Searches, Tags and Groups}
Filters, tags and groups form the basis of finding and organising objects in the user interface.

A group is an user-defined entity that references a set of other objects. It is not a container, as objects are not limited to being stored in one group (they can be referenced by many groups). Groups can be referenced by each other, but it is just a reference, and not a "nesting" behaviour. The result is a non-hierarchical graph with lines pointing from a group to another object or group.

Search is a feature provided in the environment that is used to locate objects and views. A search acts like a filter on the objects in the environment. Its results are shown as a set of objects that match the filter criteria in the environment.

Search is provided by the Object Manager.  It appears in the project overlay, already filtered to the current project, or in an Object Manager view that the user activated in their project.

A search may be constructed on the following aspects of objects:
\begin{itemize}
\item Type
\item Last Access Date
\item Keyword search on Contents
\item Location (which object database it appears in)
\end{itemize}

A tag is a saved search. An instance of Object Manager may be created with a number of pre-defined tags that search on (usually) one aspect of objects, such as their type or last access date.

\subsection{Types}

Each object has a type. The concept of "type" can be very specific, and refer to a template of an object, upto very general type definitions, such as that of "Music" or "Documents". Types are primarily used to identify which service should be used to open up a root object, display a view or instantiate in a EtoileUI controller.

The assortment of types includes:
\begin {description}
\item[Object Types] Each class defines an object type that is defined by the class that created the object.
\item[Aggregate Types] An aggregate type is a general type like "Document" or "Music" that contains a specific group of other more specific types. For example, an aggregate type "Document" would contain a list of all the "Document" types that are available in the system.
\item[Template Sub Types] Users/Developers can create new types out of object types that consist of an instance of that object type which is used as a "template" for a more specific custom  type. For example, a user may setup a word processor document with a set of specific styles and page watermarks that is intended for use as a template. It could be defined as a template type in the environment.
\end{description}

\section{Private Services}

The services within \proman and within the \etoile environment which are not directly accessible to the user are private or system services. The details of some of these are given in the following subsections

\subsection{Internal structure of services}

A service is a \gnustep tool or application. The main difference between a \gnustep application and an \etoile service is that users don't invoke services directly. 

Each service is bound to a name and one more data types. For example:
\begin{verbatim}
	'spellchecker' bound to /path/to/a/spellchecker.bundle and public.text 
	'musicmanager' bound to /path/to/melodie.app and org.etoile-project.music-library
\end{verbatim}
\subsection{Object Manager}

This service helps the user manage the objects in their workspace.

The object manager can filter the view it displays through a combination of tags and filters. 

\subsection{Workspace Manager}

This service manages the user's workspace. It is used to manage projects and views and create new objects.

It displays a bar for selecting and managing projects and switching between tags. It displays another bar that allows restroring minimised views.

\chapter{Implementation}
\section{Services}
We can track the lifetime of a service using NSTask.

For the instances where we need to start up a user-interactive service we can track the process of the service starting up to ensure that it doesn't die before it renders a view. Furthermore, we can check if a process is hung or dead by using Distributed Objects to ping or query the process upon startup and then periodically during its lifetime.
\section{X11 Integration}
\label{sec:x11_integration}
\subsection{Events}
The events listed below are used by the application portion of \proman to determine when views are created, restored, minimised and closed.
\subsubsection{MapWindow}
This event is used to track when a view is restored or newly created. The window can be identified from its window properties.
\subsubsection{UnmapWindow}
This event is used to track when a view is minimised or closed. We can distinguish between a minimise and a close by inspecting the WM\_STATE property that is set by the window manager portion.

\subsection{Window Properties}
\subsubsection{\_\_ETOILE\_VIEW\_ID}
A string containing the view identifier as created by \proman and given to the service when asking it to restore the view. 

This property is placed on a view's top-level window by the service before it is mapped. It is not updated by the service while the window is still mapped.

\subsubsection{\_\_ETOILE\_PROJECT\_ID}
A string containing the identifier of the project. It is placed on a view's top-level window by a service to indicate the project that a view belongs to. This property may be updated by a service to indicate that the view belongs to a new project, which means the window manager should monitor this property and only display the view window in the currently set project.

The \proman will place this property on the root window of the workspace to indicate the current project. When it changes this property, the window manager should re-map the windows on the display so that only the views of the newly selected project are shown.

\subsection{Messages}

\section{Core Object}
Core Object is the persistence framework in \etoile. It is used by \proman to locate objects and store information about the project workspace. Each object in a user's workspace is stored in a CoreObject database.

The CoreObject database needs to support \proman functions such as:
\begin{itemize}
\item Searching for objects
\item Storing information about the user's workspace, such as views and projects
\item Finding template (user-defined) types
\end{itemize}

Each document in \etoile is stored by CoreObject as an object graph. It has history that records each and every change that is made to the object, including to objects inside that object tree.  The document is versioned as a whole, and each document has its own revision history.

An object inside a document can refer to another object at the root or inside another document. This link does not mean that the object in the referred tree becomes a part of the history of the referring tree. It's just a pointer to the other object inside the separate document tree. The user interface must make clear when a object is a reference to another document, not an object inside the current document.

Every object inside a document object tree has its own identifier that uniquely identifies it. As all the objects are stored together in the same database, it is important that we can distinguish the roots of an object graph that forms a document. The identity of an object helps with comparing revisions in order to perform difference and merging operations, and it is needed in order to link from an object in one document to an object in another document.

\end{document}  

