\section{RSSFeed Class Reference}
\label{classRSSFeed}\index{RSSFeed@{RSSFeed}}
{\tt \#import $<$RSSFeed.h$>$}

\subsection*{Public Member Functions}
\begin{CompactItemize}
\item 
(id) - {\bf init\-With\-URL:}
\item 
(NSString $\ast$) - {\bf description}
\item 
(enum RSSFeed\-Status) - {\bf status}
\item 
({\bf RSSArticle} $\ast$) - {\bf article\-At\-Index:}
\item 
(unsigned int) - {\bf count}
\item 
(NSEnumerator $\ast$) - {\bf article\-Enumerator}
\item 
(void) - {\bf remove\-Article:}
\item 
(NSString $\ast$) - {\bf feed\-Name}
\item 
(NSURL $\ast$) - {\bf feed\-URL}
\item 
(void) - {\bf set\-Auto\-Clear:}
\item 
(BOOL) - {\bf auto\-Clear}
\item 
(void) - {\bf clear\-Articles}
\item 
(void) - {\bf set\-Article\-Class:}
\item 
(Class) - {\bf article\-Class}
\item 
(NSDate $\ast$) - {\bf last\-Retrieval}
\item 
(void) - {\bf new\-Article\-Found:}
\end{CompactItemize}
\subsection*{Protected Attributes}
\begin{CompactItemize}
\item 
enum RSSFeed\-Error {\bf last\-Error}
\item 
Class {\bf article\-Class}
\end{CompactItemize}


\subsection{Detailed Description}
Objects of this class represent a RSS/ATOM feed, which is basically just a source for new articles. When creating a {\bf RSSFeed}{\rm (p.\,\pageref{classRSSFeed})} object, you'll just have to provide it with the URL, where the feed can be downloaded from.

This is the generic way to read feeds:

\begin{itemize}
\item Create a URL object with the location of the feed.\par
 {\tt  NSURL$\ast$ url = [NSURL URLWith\-String:@\char`\"{}http://www.example.com/feed.xml\char`\"{}]; }  \item Create a {\bf RSSFeed}{\rm (p.\,\pageref{classRSSFeed})} object with the URL:\par
 {\tt  RSSFeed$\ast$ feed = [{\bf RSSFeed}{\rm (p.\,\pageref{classRSSFeed})} init\-With\-URL: url]; }  \item Fetch the contents of the feed:\par
 {\tt  enum RSSFeed\-Error err = [feed fetch]; }  \item Optionally tell the {\bf RSSFeed}{\rm (p.\,\pageref{classRSSFeed})} to keep old articles.\par
 {\tt  [feed set\-Auto\-Clear: NO]; }  \item Iterate over the articles contained in the feed. This is analogous to the iteration over a NSArray. \small\begin{alltt}
      int i;
      for (i=0; i<[feed count]; i++)
        \{
          RSSArticle* myArticle = [feed articleAtIndex: i];
          // [...] more code here
        \}
    \end{alltt}\normalsize 
  \end{itemize}


\begin{Desc}
\item[See also:]{\bf - init\-With\-URL: (RSSFeed)}{\rm (p.\,\pageref{classRSSFeed_c903aac5615dd840869a78c0f599ba4f})} 

{\bf - fetch (RSSFeed(Fetching))}{\rm (p.\,\pageref{categoryRSSFeed(Fetching)_34234d6ccb4ece076ae6ec3d86fb7f46})} 

{\bf - set\-Auto\-Clear: (RSSFeed)}{\rm (p.\,\pageref{classRSSFeed_2931b0bdee68292f638f0bc500e1662e})}

{\bf RSSArticle}{\rm (p.\,\pageref{interfaceRSSArticle})} 

NSURL \end{Desc}




\subsection{Member Function Documentation}
\index{RSSFeed@{RSSFeed}!articleAtIndex:@{articleAtIndex:}}
\index{articleAtIndex:@{articleAtIndex:}!RSSFeed@{RSSFeed}}
\subsubsection{\setlength{\rightskip}{0pt plus 5cm}- ({\bf RSSArticle} $\ast$) article\-At\-Index: (int) {\em index}}\label{classRSSFeed_67504aa7863d4037014082024212dcce}


Lets you access the individual articles in the feed. Often used in conjunction with the count method. Also take a look at the example in the description of this class. ({\bf RSSFeed}{\rm (p.\,\pageref{classRSSFeed})})

\begin{Desc}
\item[Parameters:]
\begin{description}
\item[{\em index}]of the article to get \end{description}
\end{Desc}
\begin{Desc}
\item[Returns:]Article number index \end{Desc}
\begin{Desc}
\item[See also:]{\bf RSSArticle}{\rm (p.\,\pageref{interfaceRSSArticle})} 

{\bf - count}{\rm (p.\,\pageref{classRSSFeed_fe1812219c87bda10c4ae3b2833732eb})} 

{\bf RSSFeed}{\rm (p.\,\pageref{classRSSFeed})} \end{Desc}
\index{RSSFeed@{RSSFeed}!articleClass@{articleClass}}
\index{articleClass@{articleClass}!RSSFeed@{RSSFeed}}
\subsubsection{\setlength{\rightskip}{0pt plus 5cm}- (Class) {\bf article\-Class} }\label{classRSSFeed_2dae4f34cd0cf9880881daa45cec641b}


Returns the class of the article objects. This will be a subtype of {\bf RSSArticle}{\rm (p.\,\pageref{interfaceRSSArticle})}.

\begin{Desc}
\item[Returns:]the article class \end{Desc}
\index{RSSFeed@{RSSFeed}!articleEnumerator@{articleEnumerator}}
\index{articleEnumerator@{articleEnumerator}!RSSFeed@{RSSFeed}}
\subsubsection{\setlength{\rightskip}{0pt plus 5cm}- (NSEnumerator $\ast$) article\-Enumerator }\label{classRSSFeed_aa2900034a378b5380664cec2c09cddf}


\begin{Desc}
\item[Returns:]an enumerator for the articles in this feed \end{Desc}
\index{RSSFeed@{RSSFeed}!autoClear@{autoClear}}
\index{autoClear@{autoClear}!RSSFeed@{RSSFeed}}
\subsubsection{\setlength{\rightskip}{0pt plus 5cm}- (BOOL) auto\-Clear }\label{classRSSFeed_30139f9edd798429ef3cef520167a651}


\begin{Desc}
\item[Returns:]YES, if the automatic clearing of the article list is enabled for this feed. NO otherwise. \end{Desc}
\index{RSSFeed@{RSSFeed}!clearArticles@{clearArticles}}
\index{clearArticles@{clearArticles}!RSSFeed@{RSSFeed}}
\subsubsection{\setlength{\rightskip}{0pt plus 5cm}- (void) clear\-Articles }\label{classRSSFeed_add6f191f06065bbbc7d96601cf20c8f}


Clears the list of articles. \index{RSSFeed@{RSSFeed}!count@{count}}
\index{count@{count}!RSSFeed@{RSSFeed}}
\subsubsection{\setlength{\rightskip}{0pt plus 5cm}- (unsigned int) count }\label{classRSSFeed_fe1812219c87bda10c4ae3b2833732eb}


\begin{Desc}
\item[Returns:]the number of articles in this feed. \end{Desc}
\index{RSSFeed@{RSSFeed}!description@{description}}
\index{description@{description}!RSSFeed@{RSSFeed}}
\subsubsection{\setlength{\rightskip}{0pt plus 5cm}- (NSString $\ast$) description }\label{classRSSFeed_599797aee83058da556e71125ffde55d}


\begin{Desc}
\item[Returns:]Description of the Feed (the feed name) \end{Desc}
\index{RSSFeed@{RSSFeed}!feedName@{feedName}}
\index{feedName@{feedName}!RSSFeed@{RSSFeed}}
\subsubsection{\setlength{\rightskip}{0pt plus 5cm}- (NSString$\ast$) feed\-Name }\label{classRSSFeed_b633045a0af241daedc5fc5c1aab72ea}


\begin{Desc}
\item[Returns:]The name of the feed \end{Desc}
\index{RSSFeed@{RSSFeed}!feedURL@{feedURL}}
\index{feedURL@{feedURL}!RSSFeed@{RSSFeed}}
\subsubsection{\setlength{\rightskip}{0pt plus 5cm}- (NSURL$\ast$) feed\-URL }\label{classRSSFeed_7faf1b2a0a0a5c4c6baad94f2e394834}


\begin{Desc}
\item[Returns:]the URL where the feed can be downloaded from (as NSURL object) \end{Desc}
\begin{Desc}
\item[See also:]NSURL \end{Desc}
\index{RSSFeed@{RSSFeed}!initWithURL:@{initWithURL:}}
\index{initWithURL:@{initWithURL:}!RSSFeed@{RSSFeed}}
\subsubsection{\setlength{\rightskip}{0pt plus 5cm}- (id) init\-With\-URL: (NSURL $\ast$) {\em a\-URL}}\label{classRSSFeed_c903aac5615dd840869a78c0f599ba4f}


Designated initializer.

\begin{Desc}
\item[Parameters:]
\begin{description}
\item[{\em a\-URL}]The URL where the feed can be downloaded from. \end{description}
\end{Desc}
\index{RSSFeed@{RSSFeed}!lastRetrieval@{lastRetrieval}}
\index{lastRetrieval@{lastRetrieval}!RSSFeed@{RSSFeed}}
\subsubsection{\setlength{\rightskip}{0pt plus 5cm}- (NSDate$\ast$) last\-Retrieval }\label{classRSSFeed_190fafaff71f21de561ba30be376d223}


Returns the date of last retrieval of this feed. If the feed hasn't been retrieved yet, this method returns nil.

\begin{Desc}
\item[Returns:]The date of last retrieval as a NSDate pointer. \end{Desc}
\index{RSSFeed@{RSSFeed}!newArticleFound:@{newArticleFound:}}
\index{newArticleFound:@{newArticleFound:}!RSSFeed@{RSSFeed}}
\subsubsection{\setlength{\rightskip}{0pt plus 5cm}- (void) new\-Article\-Found: ({\bf RSSArticle} $\ast$) {\em an\-Article}}\label{classRSSFeed_f3f84cc415099c96c1b138cc8d0aa285}


{\bf RSSFeed}{\rm (p.\,\pageref{classRSSFeed})} also implements the New\-RSSArticle\-Listener informal protocol. \index{RSSFeed@{RSSFeed}!removeArticle:@{removeArticle:}}
\index{removeArticle:@{removeArticle:}!RSSFeed@{RSSFeed}}
\subsubsection{\setlength{\rightskip}{0pt plus 5cm}- (void) remove\-Article: ({\bf RSSArticle} $\ast$) {\em article}}\label{classRSSFeed_29afdf6eb5be55fabfe9aed791194ed2}


Deletes an article from the feed.

\begin{Desc}
\item[Parameters:]
\begin{description}
\item[{\em article}]The index of the article to delete. \end{description}
\end{Desc}
\index{RSSFeed@{RSSFeed}!setArticleClass:@{setArticleClass:}}
\index{setArticleClass:@{setArticleClass:}!RSSFeed@{RSSFeed}}
\subsubsection{\setlength{\rightskip}{0pt plus 5cm}- (void) set\-Article\-Class: (Class) {\em a\-Class}}\label{classRSSFeed_88a649a3922cc4c629212bb6f53a716f}


Sets the class of the article objects. This needs to be a subtype of {\bf RSSArticle}{\rm (p.\,\pageref{interfaceRSSArticle})}.

\begin{Desc}
\item[Parameters:]
\begin{description}
\item[{\em a\-Class}]The class newly created article objects should have. \end{description}
\end{Desc}
\index{RSSFeed@{RSSFeed}!setAutoClear:@{setAutoClear:}}
\index{setAutoClear:@{setAutoClear:}!RSSFeed@{RSSFeed}}
\subsubsection{\setlength{\rightskip}{0pt plus 5cm}- (void) set\-Auto\-Clear: (BOOL) {\em auto\-Clear}}\label{classRSSFeed_2931b0bdee68292f638f0bc500e1662e}


Lets you decide if the feed should be cleared before new articles are downloaded.

\begin{Desc}
\item[Parameters:]
\begin{description}
\item[{\em auto\-Clear}]YES, if the feed should clear its article list before fetching new articles. NO otherwise \end{description}
\end{Desc}
\index{RSSFeed@{RSSFeed}!status@{status}}
\index{status@{status}!RSSFeed@{RSSFeed}}
\subsubsection{\setlength{\rightskip}{0pt plus 5cm}- (enum RSSFeed\-Status) status }\label{classRSSFeed_c44486e9314ab95c8d6b7a139df1cf24}


Accessor for the status of the feed. This can be used by a multithreaded GUI to indicate if a feed is currently fetching...

\begin{Desc}
\item[Returns:]either RSSFeed\-Is\-Fetching or RSSFeed\-Is\-Idle \end{Desc}


\subsection{Member Data Documentation}
\index{RSSFeed@{RSSFeed}!articleClass@{articleClass}}
\index{articleClass@{articleClass}!RSSFeed@{RSSFeed}}
\subsubsection{\setlength{\rightskip}{0pt plus 5cm}- (Class) {\bf article\-Class}\hspace{0.3cm}{\tt  [protected]}}\label{classRSSFeed_46d6ffae81a000ce1751557cec10a1d1}


Returns the class of the article objects. This needs to be a subclass of {\bf RSSArticle}{\rm (p.\,\pageref{interfaceRSSArticle})}. (Also needed to implement the New\-RSSArticle\-Listener class)

\begin{Desc}
\item[Returns:]the article class \end{Desc}
\index{RSSFeed@{RSSFeed}!lastError@{lastError}}
\index{lastError@{lastError}!RSSFeed@{RSSFeed}}
\subsubsection{\setlength{\rightskip}{0pt plus 5cm}- (enum RSSFeed\-Error) {\bf last\-Error}\hspace{0.3cm}{\tt  [protected]}}\label{classRSSFeed_461f168ad127ec8e2b27daa991fcad41}


Returns the last error. Guaranteed to return the last fetching result. 

The documentation for this class was generated from the following files:\begin{CompactItemize}
\item 
RSSFeed+Fetching.m\item 
RSSFeed.h\item 
RSSFeed.m\end{CompactItemize}
