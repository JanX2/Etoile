\section{RSSArticle Class Reference}
\label{interfaceRSSArticle}\index{RSSArticle@{RSSArticle}}
{\tt \#import $<$RSSArticle.h$>$}

\subsection*{Public Member Functions}
\begin{CompactItemize}
\item 
(id) - {\bf init}
\item 
(id) - {\bf init\-With\-Headline:url:description:date:}
\item 
(id) - {\bf init\-With\-Headline:url:description:time:}
\item 
(id) - {\bf init\-With\-Coder:}\label{interfaceRSSArticle_6984252682d7e137046486ffff11dd49}

\begin{CompactList}\small\item\em Deserializes a {\bf RSSArticle}{\rm (p.\,\pageref{interfaceRSSArticle})} object from a NSCoder. \item\end{CompactList}\item 
(void) - {\bf encode\-With\-Coder:}\label{interfaceRSSArticle_f3209eed527bf05309bcc99c54908d4d}

\begin{CompactList}\small\item\em Serializes a {\bf RSSArticle}{\rm (p.\,\pageref{interfaceRSSArticle})} object to a NSCoder. \item\end{CompactList}\item 
(NSString $\ast$) - {\bf headline}
\item 
(NSString $\ast$) - {\bf url}
\item 
(NSString $\ast$) - {\bf description}
\item 
(void) - {\bf add\-Link:}
\item 
(void) - {\bf set\-Links:}
\item 
(NSArray $\ast$) - {\bf links}
\item 
(NSDate $\ast$) - {\bf date}
\item 
({\bf RSSFeed} $\ast$) - {\bf feed}
\item 
(BOOL) - {\bf is\-Equal:}
\end{CompactItemize}


\subsection{Detailed Description}
An object of this class represents an article in an RSS Feed. 



\subsection{Member Function Documentation}
\index{RSSArticle@{RSSArticle}!addLink:@{addLink:}}
\index{addLink:@{addLink:}!RSSArticle@{RSSArticle}}
\subsubsection{\setlength{\rightskip}{0pt plus 5cm}- (void) add\-Link: (NSURL $\ast$) {\em an\-URL}}\label{interfaceRSSArticle_739a84b52475f06f053778bebf925fb7}


Adds a new link to this article. This is a NSURL object, which usually has the \char`\"{}type\char`\"{} property set to an NSString which represents the resource's MIME type. You may also specify the \char`\"{}rel\char`\"{} property, which should be one of \char`\"{}enclosure\char`\"{}, \char`\"{}related\char`\"{}, \char`\"{}alternate\char`\"{}, \char`\"{}via\char`\"{}. \index{RSSArticle@{RSSArticle}!date@{date}}
\index{date@{date}!RSSArticle@{RSSArticle}}
\subsubsection{\setlength{\rightskip}{0pt plus 5cm}- (NSDate$\ast$) date }\label{interfaceRSSArticle_4b4aae899bd76df47234b07f3add2095}


Returns the date of the publication of the article. If the source feed of this article didn't contain information about this date, the fetching date is usually returned.

\begin{Desc}
\item[Returns:]The date of the publication of the article \end{Desc}
\index{RSSArticle@{RSSArticle}!description@{description}}
\index{description@{description}!RSSArticle@{RSSArticle}}
\subsubsection{\setlength{\rightskip}{0pt plus 5cm}- (NSString$\ast$) description }\label{interfaceRSSArticle_4288085868ea28184f9cfc6851d2c67a}


\begin{Desc}
\item[Returns:]The full text, an excerpt or a summary from the article \end{Desc}
\index{RSSArticle@{RSSArticle}!feed@{feed}}
\index{feed@{feed}!RSSArticle@{RSSArticle}}
\subsubsection{\setlength{\rightskip}{0pt plus 5cm}- ({\bf RSSFeed}$\ast$) feed }\label{interfaceRSSArticle_bf64eba4838704987e6afa3d2e2c967f}


Returns the source feed of this article.

\begin{Desc}
\item[Warning:]It's not guaranteed that this object actually exists. Be aware of segmentation faults!\end{Desc}
If you want to make sure the object exists, you have to follow these rules:

\begin{itemize}
\item Don't retain any article! \item Don't call the (undocumented) {\tt feed:} (Colon!) method. \end{itemize}


\begin{Desc}
\item[Returns:]The source feed of this article \end{Desc}
\index{RSSArticle@{RSSArticle}!headline@{headline}}
\index{headline@{headline}!RSSArticle@{RSSArticle}}
\subsubsection{\setlength{\rightskip}{0pt plus 5cm}- (NSString$\ast$) headline }\label{interfaceRSSArticle_5c0896c6f2ae76e658dccd0a7e9d37db}


\begin{Desc}
\item[Returns:]The headline of the article \end{Desc}
\index{RSSArticle@{RSSArticle}!init@{init}}
\index{init@{init}!RSSArticle@{RSSArticle}}
\subsubsection{\setlength{\rightskip}{0pt plus 5cm}- (id) init }\label{interfaceRSSArticle_8223fd4f05616bcd99f7bbbbcb281896}


Standard initializer. You shouldn't use this. Better use init\-With\-Headline:url:description:date:

\begin{Desc}
\item[See also:]{\bf - init\-With\-Headline:url:description:date:}{\rm (p.\,\pageref{interfaceRSSArticle_72252074ba9b2edc3d4b21a9e67d5492})} \end{Desc}
\index{RSSArticle@{RSSArticle}!initWithHeadline:url:description:date:@{initWithHeadline:url:description:date:}}
\index{initWithHeadline:url:description:date:@{initWithHeadline:url:description:date:}!RSSArticle@{RSSArticle}}
\subsubsection{\setlength{\rightskip}{0pt plus 5cm}- (id) init\-With\-Headline: (NSString $\ast$) {\em my\-Headline}(NSString $\ast$) {\em my\-Url}(NSString $\ast$) {\em my\-Description}(NSDate $\ast$) {\em my\-Date}}\label{interfaceRSSArticle_72252074ba9b2edc3d4b21a9e67d5492}


Designated initializer for the {\bf RSSArticle}{\rm (p.\,\pageref{interfaceRSSArticle})} class.

Don't create {\bf RSSArticle}{\rm (p.\,\pageref{interfaceRSSArticle})} objects yourself. Create a {\bf RSSFeed}{\rm (p.\,\pageref{classRSSFeed})} object and let it fetch the articles for you!

\begin{Desc}
\item[Parameters:]
\begin{description}
\item[{\em my\-Headline}]A NSString containing the headline of the article. \item[{\em my\-Url}]A NSString containing the URL of the full version of the article. \item[{\em my\-Description}]An excerpt of the article text or the full text. \item[{\em my\-Date}]The date as NSDate object on which this article was posted. \end{description}
\end{Desc}
\begin{Desc}
\item[See also:]{\bf RSSFeed}{\rm (p.\,\pageref{classRSSFeed})} \end{Desc}
\index{RSSArticle@{RSSArticle}!initWithHeadline:url:description:time:@{initWithHeadline:url:description:time:}}
\index{initWithHeadline:url:description:time:@{initWithHeadline:url:description:time:}!RSSArticle@{RSSArticle}}
\subsubsection{\setlength{\rightskip}{0pt plus 5cm}- (id) init\-With\-Headline: (NSString $\ast$) {\em my\-Headline}(NSString $\ast$) {\em my\-Url}(NSString $\ast$) {\em my\-Description}(unsigned int) {\em my\-Time}}\label{interfaceRSSArticle_2bec46ce7a4c7520d9d2b353f30c13d8}


Old designated initializer for the {\bf RSSArticle}{\rm (p.\,\pageref{interfaceRSSArticle})} class. Only here for compatibility reasons. This method is likely to be dropped in future versions. \begin{Desc}
\item[{\bf Deprecated}]\end{Desc}
\begin{Desc}
\item[Parameters:]
\begin{description}
\item[{\em my\-Headline}]A NSString containing the headline of the article. \item[{\em my\-Url}]A NSString containing the URL of the full version of the article. \item[{\em my\-Description}]An excerpt of the article text or the full text. \item[{\em my\-Time}]The date (in seconds since 1970) on which this article was posted. \end{description}
\end{Desc}
\index{RSSArticle@{RSSArticle}!isEqual:@{isEqual:}}
\index{isEqual:@{isEqual:}!RSSArticle@{RSSArticle}}
\subsubsection{\setlength{\rightskip}{0pt plus 5cm}- (BOOL) is\-Equal: (id) {\em an\-Object}}\label{interfaceRSSArticle_508fd131595c1ae980cf9e2024641ad1}


RSS Articles are equal if both the article headlines and the article URLs are equal. If they are equal is tested by calling the is\-Equal: method on those. \index{RSSArticle@{RSSArticle}!links@{links}}
\index{links@{links}!RSSArticle@{RSSArticle}}
\subsubsection{\setlength{\rightskip}{0pt plus 5cm}- (NSArray$\ast$) links }\label{interfaceRSSArticle_6f458ddf4bceb58db46ac1c6c723b337}


Returns an NSArray containing NSURL objects or nil, if there are none. The contained NSURL objects often have the \char`\"{}type\char`\"{} and \char`\"{}rel\char`\"{} properties set. See the documentation for add\-Link: for details.

\begin{Desc}
\item[Returns:]The links of the article. \end{Desc}
\index{RSSArticle@{RSSArticle}!setLinks:@{setLinks:}}
\index{setLinks:@{setLinks:}!RSSArticle@{RSSArticle}}
\subsubsection{\setlength{\rightskip}{0pt plus 5cm}- (void) set\-Links: (NSMutable\-Array $\ast$) {\em some\-Links}}\label{interfaceRSSArticle_ca5aede67d6dab975c7457e3af06fc6c}


Replaces the list of links with a new one. See the documentation for add\-Link: for details. Hint: The parameter may also be nil. \index{RSSArticle@{RSSArticle}!url@{url}}
\index{url@{url}!RSSArticle@{RSSArticle}}
\subsubsection{\setlength{\rightskip}{0pt plus 5cm}- (NSString$\ast$) url }\label{interfaceRSSArticle_e6488ed33911d3717c256058d902c477}


\begin{Desc}
\item[Returns:]Te URL of the full version of the article (as NSString$\ast$) \end{Desc}


The documentation for this class was generated from the following files:\begin{CompactItemize}
\item 
RSSArticle.h\item 
RSSArticle.m\end{CompactItemize}
