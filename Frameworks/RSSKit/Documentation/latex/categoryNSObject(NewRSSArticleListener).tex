\section{NSObject(New\-RSSArticle\-Listener) Category Reference}
\label{categoryNSObject(NewRSSArticleListener)}\index{NSObject(NewRSSArticleListener)@{NSObject(NewRSSArticleListener)}}
{\tt \#import $<$New\-RSSArticle\-Listener.h$>$}

\subsection*{Public Member Functions}
\begin{CompactItemize}
\item 
(void) - {\bf new\-Article\-Found:}
\item 
(Class) - {\bf article\-Class}
\end{CompactItemize}


\subsection{Detailed Description}
This protocol is intended to be implemented by classes which want to be notified of articles found when parsing RSS feeds. 



\subsection{Member Function Documentation}
\index{NSObject(NewRSSArticleListener)@{NSObject(New\-RSSArticle\-Listener)}!articleClass@{articleClass}}
\index{articleClass@{articleClass}!NSObject(NewRSSArticleListener)@{NSObject(New\-RSSArticle\-Listener)}}
\subsubsection{\setlength{\rightskip}{0pt plus 5cm}- (Class) article\-Class }\label{categoryNSObject(NewRSSArticleListener)_cc03170efe1b023fcd88165ecfe7f1f4}


Returns the class of the article objects. This needs to be a subclass of {\bf RSSArticle}{\rm (p.\,\pageref{interfaceRSSArticle})}.

\begin{Desc}
\item[Returns:]the article class \end{Desc}
\index{NSObject(NewRSSArticleListener)@{NSObject(New\-RSSArticle\-Listener)}!newArticleFound:@{newArticleFound:}}
\index{newArticleFound:@{newArticleFound:}!NSObject(NewRSSArticleListener)@{NSObject(New\-RSSArticle\-Listener)}}
\subsubsection{\setlength{\rightskip}{0pt plus 5cm}- (void) new\-Article\-Found: ({\bf RSSArticle} $\ast$) {\em an\-Article}}\label{categoryNSObject(NewRSSArticleListener)_1f93b7d179e71239abb907c5bff99983}


This method gets called when a new article has been found. 

The documentation for this category was generated from the following file:\begin{CompactItemize}
\item 
New\-RSSArticle\-Listener.h\end{CompactItemize}
